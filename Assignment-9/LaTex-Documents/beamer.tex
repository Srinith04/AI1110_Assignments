%%%%%%%%%%%%%%%%%%%%%%%%%%%%%%%%%%%%%%%%%%%%%%%%%%%%%%%%%%%%%%%
%
% Welcome to Overleaf --- just edit your LaTeX on the left,
% and we'll compile it for you on the right. If you open the
% 'Share' menu, you can invite other users to edit at the same
% time. See www.overleaf.com/learn for more info. Enjoy!
%
%%%%%%%%%%%%%%%%%%%%%%%%%%%%%%%%%%%%%%%%%%%%%%%%%%%%%%%%%%%%%%%

% Inbuilt themes in beamer
\documentclass{beamer}

%packages:
% \usepackage{tfrupee}
% \usepackage{amsmath}
% \usepackage{amssymb}
% \usepackage{gensymb}
% \usepackage{txfonts}

% \def\inputGnumericTable{}

% \usepackage[latin1]{inputenc}                                 
% \usepackage{color}                                            
% \usepackage{array}                                            
% \usepackage{longtable}                                        
% \usepackage{calc}                                             
% \usepackage{multirow}                                         
% \usepackage{hhline}                                           
% \usepackage{ifthen}
% \usepackage{caption} 
% \captionsetup[table]{skip=3pt}  
% \providecommand{\pr}[1]{\ensuremath{\Pr\left(#1\right)}}
% \providecommand{\cbrak}[1]{\ensuremath{\left\{#1\right\}}}
% %\renewcommand{\thefigure}{\arabic{table}}
% \renewcommand{\thetable}{\arabic{table}}      

\setbeamertemplate{caption}[numbered]{}

\usepackage{enumitem}
\usepackage{tfrupee}
\usepackage{amsmath}
\usepackage{amssymb}
\usepackage{graphicx}
\usepackage{txfonts}

\def\inputGnumericTable{}

\usepackage[latin1]{inputenc}                                 
\usepackage{color}                                            
\usepackage{array}                                            
\usepackage{longtable}                                        
\usepackage{calc}                                             
\usepackage{multirow}                                         
\usepackage{hhline}                                           
\usepackage{ifthen}
\usepackage{caption} 
\captionsetup[table]{skip=3pt}  
\providecommand{\pr}[1]{\ensuremath{\Pr\left(#1\right)}}
\providecommand{\cbrak}[1]{\ensuremath{\left\{#1\right\}}}
\renewcommand{\thefigure}{\arabic{table}}
\renewcommand{\thetable}{\arabic{table}}   
\providecommand{\brak}[1]{\ensuremath{\left(#1\right)}}

% Theme choice:
\usetheme{CambridgeUS}

% Title page details: 
\title{Assignment 9} 
\author{Dasari Srinith}
\date{\today}
% \logo{\large \LaTeX{}}


\begin{document}

    % Title page
    \begin{frame}
        \titlepage 
    \end{frame}

    % Outline
    \begin{frame}{Outline}
        \tableofcontents
    \end{frame}

    \section{Question}
    	\begin{frame}{Exercise 13.3.5}
    	A laboratory blood test is 99\% effective in detecting a certain disease when it is in fact, present. However, the test also yields a false positive result for 0.5\% of the healthy person tested (i.e. if a healthy person is tested, then, with probability 0.005, the test will imply he has the disease). If 0.1 percent of the population actually has the disease, what is the probability that a person has the disease given that his test result is positive ?
    	\end{frame}

    \section{Theory}
        \begin{frame}{Bayes' Theorem}
        Bayes theorem is used to determine conditional probability.
        \begin{block}{}
        \begin{align}
            \pr{(E_i|A)} &= \dfrac{\pr{(AE_i)}}{\pr{A}} \\[10pt]
                         &= \dfrac{\pr{(AE_i)}}{\sum_{i=0}^n \pr{AE_i}} 
                         \label{eq:2}
        \end{align}
        \end{block}
        \end{frame}

    \section{Solution}
    \begin{frame}{Solution}
    Let $X , Y$ be two random variables which maps to following set of real numbers , $X \in \cbrak {0,1}$ , $Y \in \cbrak {0,1}$ as defined below in Table \ref{Table1} , Table \ref{Table2} respectively.
    \begin{table}[ht!]
        \input{Tables/TABLE1.tex}
        \caption{Defining the events for $X$}
        \label{Table1}
    \end{table}
    \end{frame}
    
    \begin{frame}{}
    \begin{table}[ht!]
        \input{Tables/TABLE2.tex}
        \caption{Defining the events for $Y$}
        \label{Table2}
    \end{table}
    We have to find $\pr{(X=1)|(Y=1)}$.
    \end{frame}
    
    \begin{frame}{Input data}
    From the data given in the question we can write the following ,
    \begin{table}[ht!]
        \input{Tables/TABLE3.tex}
        \caption{Input data}
        \label{Table3}
    \end{table}
    \end{frame}
    
    \begin{frame}{Computation}
    Since the events $X=0$ and $X=1$ are complimentary to each other ,
    \begin{align}
        \pr{(X=0)} + \pr{(X=1)} &= 1 \\
        \pr{(X=0)} &= 0.999
    \end{align} 
    \end{frame}
    
    \begin{frame}{}
        By using Bayes' theorem i.e, \eqref{eq:2} we have
    \begin{align}
        &\pr{(X=1)|(Y=1)} \nonumber = \\
        &\dfrac{\pr{(X=1)} \pr{(Y=1)|(X=1)}}{\pr{(X=1)} \pr{(Y=1)|(X=1)}+\pr{(X=0)} \pr{(Y=1)|(X=0)}} \\[10pt]
         &= \dfrac{(0.001)(0.99)}{(0.001)(0.99) + (0.999)(0.005)} \\[10pt]
         &= \dfrac{0.00099}{0.005985} \\[10pt]
         &= \dfrac{990}{5985} = \dfrac{22}{133}
    \end{align}
    \end{frame}
    
    \section{Result}
    \begin{frame}{Result}
        If 0.1 percent of the population actually has the disease , the probability that a person has the disease given that his test result is positive is $\dfrac{22}{133}$
    \end{frame}

\end{document}
