%%%%%%%%%%%%%%%%%%%%%%%%%%%%%%%%%%%%%%%%%%%%%%%%%%%%%%%%%%%%%%%
%
% Welcome to Overleaf --- just edit your LaTeX on the left,
% and we'll compile it for you on the right. If you open the
% 'Share' menu, you can invite other users to edit at the same
% time. See www.overleaf.com/learn for more info. Enjoy!
%
%%%%%%%%%%%%%%%%%%%%%%%%%%%%%%%%%%%%%%%%%%%%%%%%%%%%%%%%%%%%%%%

% Inbuilt themes in beamer
\documentclass{beamer}

%packages:
% \usepackage{tfrupee}
% \usepackage{amsmath}
% \usepackage{amssymb}
% \usepackage{gensymb}
% \usepackage{txfonts}

% \def\inputGnumericTable{}

% \usepackage[latin1]{inputenc}                                 
% \usepackage{color}                                            
% \usepackage{array}                                            
% \usepackage{longtable}                                        
% \usepackage{calc}                                             
% \usepackage{multirow}                                         
% \usepackage{hhline}                                           
% \usepackage{ifthen}
% \usepackage{caption} 
% \captionsetup[table]{skip=3pt}  
% \providecommand{\pr}[1]{\ensuremath{\Pr\left(#1\right)}}
% \providecommand{\cbrak}[1]{\ensuremath{\left\{#1\right\}}}
% %\renewcommand{\thefigure}{\arabic{table}}
% \renewcommand{\thetable}{\arabic{table}}      

\setbeamertemplate{caption}[numbered]{}

\usepackage{enumitem}
\usepackage{tfrupee}
\usepackage{amsmath}
\usepackage{amssymb}
\usepackage{graphicx}
\usepackage{txfonts}

\def\inputGnumericTable{}

\usepackage[latin1]{inputenc}                                 
\usepackage{color}                                            
\usepackage{array}                                            
\usepackage{longtable}                                        
\usepackage{calc}                                             
\usepackage{multirow}                                         
\usepackage{hhline}                                           
\usepackage{ifthen}
\usepackage{caption} 
\captionsetup[table]{skip=3pt}  
\providecommand{\pr}[1]{\ensuremath{\Pr\left(#1\right)}}
\providecommand{\cbrak}[1]{\ensuremath{\left\{#1\right\}}}
\renewcommand{\thefigure}{\arabic{table}}
\renewcommand{\thetable}{\arabic{table}}   
\newcommand*{\Comb}[2]{{}^{#1}C_{#2}}
\providecommand{\brak}[1]{\ensuremath{\left(#1\right)}}

% Theme choice:
\usetheme{CambridgeUS}

% Title page details: 
\title{Assignment 10} 
\author{Dasari Srinith}
\date{\today}
% \logo{\large \LaTeX{}}


\begin{document}

    % Title page
    \begin{frame}
        \titlepage 
    \end{frame}

    % Outline
    \begin{frame}{Outline}
        \tableofcontents
    \end{frame}

    \section{Question}
    	\begin{frame}{Example 34}
    	Find the mean of the binomial distribution B \brak{4,\dfrac{1}{3}}
    	\end{frame}

    \section{Theory}
        \begin{frame}{Theory}
        A binomial distribution with n- Bernoulli trials and probability of success in each trial as p , is denoted by B \brak{n,p}
        \\\vspace{0.5cm}
        The probability of k successes $\pr{X = k}$ is also denoted by $P(k)$ and is given by 
            \begin{align}
                \pr{X = k} = \Comb{n}{k} p^k\brak{1-p}^{n-k}
                \label{eq:1}
            \end{align}
            for $x = 0,1,2,...,n-1,n$ 
        \end{frame}
        
        \begin{frame}{Mean}
        \begin{align}
            \mu &= \sum^{n}_{i=1} x_i P(x_i) \\
            \mu &= \sum^{n}_{r=1} r \times \Comb{n}{r} p^r\brak{1-p}^{n-r} \\
            \mu &= (1-p)^{n} \sum^{n}_{r=1} r \times                 \Comb{n}{r}\brak{\dfrac{p}{1-p}}^r \\
            \mu &= (1-p)^{n} \dfrac{np}{(1-p)^{n}} \\
            \mu &= np
            \label{eq:6}
        \end{align}
            
        \end{frame}
    
    \section{Solution}
    \begin{frame}{Solution}
    Let $X$ be the random variable whose probability distribution is B \brak{4,\dfrac{1}{3}}.
    
    So , we can write that ,
    \begin{align}
        n &=4 \\
        p &= \dfrac{1}{3} \\
        q &= 1- p = \dfrac{2}{3}
    \end{align}
    From \eqref{eq:1} we can say,
    \begin{align}
        \pr{X = k} = \Comb{4}{k} \brak{\dfrac{1}{3}}^k\brak{\dfrac{2}{3}}^{4-k}
    \end{align}
    for $k = 0,1,2,3,4$
    \end{frame}
    
    \begin{frame}{Distribution of X}
    \begin{table}[ht!]
    \def\arraystretch{1.53}
        \input{Tables/TABLE1.tex}
        \caption{Probability Distribution of X}
        \label{Table1}
    \end{table}
    \end{frame}
    
    \begin{frame}{Mean ($\mu$)}
    We know that, from \eqref{eq:6}
        \begin{align}
            \mu &= np \\
            \mu &= \dfrac{4}{3}
        \end{align}
    \end{frame}
    
    \section{Result}
    \begin{frame}{Result}
        The mean of the binomial distribution B $\brak{4,\dfrac{1}{3}} = \dfrac{4}{3}$
    \end{frame}

\end{document}


